\documentclass[11pt]{article}

\usepackage[T1]{fontenc}
\usepackage[utf8]{inputenc}
\usepackage[spanish]{babel}
\usepackage{amsmath}

\newcommand\LL{\mathcal{L}}


\title{Intervalos de confianza: MonteCarlo vs. Bootstrap}
\author{Felipe Gerard}
\date{Octubre de 2015}

\begin{document}

\maketitle

\section*{Antecedentes}

Cuando se estima una cantidad de interés, es importante conocer la precisión del estimador que se está utilizando. Cuando la distribución es conocida (o parecida a una conocida), se puede hacer supuestos y calcular explícitamente la distribución del estimador, de donde se puede obtener intervalos de confianza. Sin embargo, cuando es desconocida, usualmente se recurre a propiedades asintóticas de los estimadores. En particular, se supone que el estimador ``ya convergió'' a una normal (por el TCL) y se utilizan los intervalos de confianza asociados a una normal. La pregunta es entonces si esto es una buena idea o si hay algún método para obtener intervalos de mejor calidad. En este trabajo compararemos los intervalos asintóticos contra los obtenidos con remuestreo bootstrap.

\section*{Planteamiento}

Supongamos que tenemos una variable aleatoria $X \sim f$, de la cual queremos conocer $\theta = E[\varphi(X)]$. Supongamos que tenemos una muestra aleatoria $\LL = \{X_1, \dots, X_N\}$ de $X$ y que vamos a estimar $\theta$ con $\hat\theta = \hat\theta(\LL)$. Aludiendo a Montecarlo, tendríamos
\[
\hat\theta = \frac{1}{N} \sum_{i=1}^N \varphi(X_i)
	\rightarrow N(\theta, \sigma_{\varphi(X)}^2/N)
\]
De este modo, si estimamos $\theta$ con $\hat\theta$ y $\sigma_{\varphi(X)}^2$ con la varianza muestral $S_{\varphi(X)}^2$, entonces el intervalo de confianza asintótico aproximado para $\theta$ es
\[
\hat\theta \pm S_{\varphi(X)}/\sqrt{N}
\]
Ahora bien, otra forma de obtener intervalos es utilizando remuestreo bootstrap. Para ello, en lugar de utilizar una muestra real de $X$ pero con una distribución aproximadamente normal para $\hat\theta$, el enfoque es obtener una muestra aproximada de $\hat\theta$ y utilizar la distribución real (de la muestra aproximada) para obtener los intervalos. Una remuestra bootstrap de $\LL$ es una muestra con reemplazo de tamaño $N$, $\LL_b = \{X^b_1, \dots, X^b_N\}$, de la cual podemos obtener
\[
\hat\theta_b = \frac{1}{N} \sum_{i=1}^N \varphi(X^b_i),
\]
que se distribuye aproximadamente como $\hat\theta$. Si obtenemos $B$ remuestras bootstrap, tendremos una muestra aleatoria $\hat\LL_\theta = \{\hat\theta_1, \dots, \hat\theta_B\}$ que aproxima la verdadera distribución de $\hat\theta$. Podemos obtener por ejemplo los intervalos de confianza empíricos simples con significancia $1 - \alpha$:
\[
[q^B_{\alpha/2},\; q^B_{1 - \alpha/2}]
\]

\section*{Comparación teórica}















\section*{Bibliografía}

DiCiccio, Thomas J.; Efron, Bradley. Bootstrap confidence intervals. 
 Statist. Sci. 11 (1996), no. 3, 189--228.










\end{document}