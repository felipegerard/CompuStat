\documentclass[11pt]{article}

\usepackage[T1]{fontenc}
\usepackage[utf8]{inputenc}
\usepackage[spanish]{babel}


\title{Intervalos de confianza: MonteCarlo vs. Bootstrap}
\author{Felipe Gerard}
\date{Octubre de 2015}

\begin{document}

\maketitle

\section*{Antecedentes}

Cuando se estima una cantidad de interés, es importante conocer la precisión del estimador que se está utilizando. Supongamos que tenemos una variable aleatoria $X \sim f$ y queremos conocer $\theta = E[\varphi(X)]$. Dependiendo de la información con la que contemos, podemos recurrir a diversos métodos. En particular, si conocemos $f$ y podemos simular de ella, podemos utilizar MonteCarlo. Si, por otra parte, no conocemos $f$ pero tenemos una muestra aleatoria $X_1, \dots, X_n$ de $f$, entonces podemos utilizar bootstrap.

En ambos métodos se puede calcular intervalos de confianza para las estimaciones. En este trabajo intentaremos dar información sobre la calidad de ambas.



























\end{document}